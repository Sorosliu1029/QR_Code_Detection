\section{实验和讨论}
\subsection{实验设置}
实验的方式主要采用的是绘制出检测框,作人眼的评价,同时监测算法执行的时间。
\subsection{评价指标}
(1).二维码检测的准确率:在原始图像上绘制出检测方法的结果,评价检测框和真实二维码位置的重叠率。\\
(2).二维码检测的效率:监测二维码检测算法执行的时间
\subsection{实验结果}
(1).在准确率上,基本方法只能识别单个方正的二维码。对于多个二维码的情况,则全部误检。
而改进方法能够识别所有的测试图像,包括多个二维码并存,或残缺二维码的图像,以及变形的图像。\\
(2).在效率上,对于同一张二维码图像,基本方法大致比改进方法快了三倍。
\subsection{比较和讨论}
从上述实验的结果看,大致和我们在方法优缺点比较时的结论一致。\\
从准确度上讲,改进方法针对基本方法在检测上的不足做了专门的优化,检测的性能必然比基本方法好,能够处理的复杂情况也比基本方法多。
另一方面,正是因为改进方法增强了检测能力,相应地就要付出检测时间上的代价,比基本方法慢两三倍也是很正常的。总的来说,即使有时间上的额外开销,改进算法也仍能满足实时性的要求。