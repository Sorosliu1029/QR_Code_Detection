\section{引言}
\subsection{问题陈述}
本报告主要依据二维码的编码特点,研究二维码的检测方法。
虽然二维码的编码规范有助于实现快速检测的算法,但是真实场景中,二维码取景时往往伴随着光照不足,摄像设备分辨率低,图像倾斜,变形,扭曲,甚至部分被遮挡的情况。
而我们不单单满足于规范二维码的检测,而是要挑战复杂摄像条件下二维码的检测。\\
具体挑战分为:1. 光照不足引起的图像二值化效果不理想问题,2.摄像设备分辨率低引起的图像边缘检测问题,3.二维码取景时的噪声问题,4.二维码定位模块的位置检测问题,5.多二维码分割问题,6.变形图像的变换问题。

\subsection{应用场合}
由于二维码具有信息编码密度高,信息类型丰富,版本众多,易于定制化等等特点,二维码在生活生产中具有广泛的应用场景。而实现在复杂场景下的二维码检测方法,则能够帮助二维码应用到更多场合中,如光照不足的矿业,使用低端摄像设备的流水线生产行业等等。

\subsection{研究历史}
针对二维码检测中的主要环节,我们依次介绍相关方面的研究历史。

\subsubsection{二维码图像的二值化问题}
应用场景中采集的二维码图像是RGB模式的三色图像,为了减少之后图像处理的计算量,加速二维码检测算法,我们首先需要将二维码图像转化为灰度图。
摄像设备在不同光照条件下采集到的图像灰度水平不同。对于二维码来说,明确地分辨出对应像素点是黑还是白,是便于之后二维码定位的关键步骤。因此能否有效地将二维码二值化是一个很重要的问题。\\
对图像进行二值化分为全局阈值方法和局部阈值方法。\\
全局阈值方法中最简单最通用的方法是固定阈值,将某一灰度以下的像素全部设为黑色,而其他像素设为白色。
而另一种效果较好的方法是找出图像中的最大值和最小值,然后将中点作为阈值。典型的方法是OTSU算法(最大类间方差法)。OTSU的中心思想是最好的阈值应该使得被阈值分开的两组的方差加权和达到最大\cite{1}。而Kittler的实验\cite{2}也表明,当图像灰度直方图呈现双峰时,使用最大类间方差法效果较好,而当图像中目标与背景的大小之比较小时,该方法的效果较差。\\
Bernsen是局部阈值方法的典型算法。它通过定义考察点的邻域,并由邻域计算模板来实现考察点灰度与邻域的比较,它较适合解决光照不均的问题,而影响二维码图像读取的最严重问题也正是光照不均问题,但Bernsen算法也有速度慢,对干扰比较敏感和易导致伪影的缺点,特别是伪影问题,如果伪影现象严重,不仅会给图像带来大量的干扰,严重的还会导致后续译码的失败\cite{3}\cite{4}\cite{5}。这是因为Bernsen算法以局部窗口内最大、最小值作为考察点的域值。当局部窗口内无目标经过时,个别噪声点就会引起阈值的剧烈变化。另外,背景灰度的非均匀性也可能影响到局部阈值。\\
Shih-Chang Hsia等人提出一种自适应的补偿背景光的方法\cite{6},可以将其用于二维码二值化前的预处理。

\subsubsection{倾斜纠正}
倾斜的二维码需要倾斜纠正。根据图像边缘可以获得二维码的倾斜角度,然后将图像旋正。边缘是图像最基本的特征,所谓边缘是指图像周围像素灰度有阶跃变化或屋顶状变化的像素的集合。经典的边缘检测算法是构造对像素灰度级阶跃变化敏感的微分算子,常用的边缘提取算子有Sobel边缘算子,Laplacian算子,Canny算子等等\cite{7}。边缘检测算法得到的往往是断续的、不完整的结构信息,对噪声较为敏感。为了有效抑制噪声,在作边缘检测前,必须对图像进行平滑处理。另外,使用微分算子进行边缘检测的计算量仍然很大。\\
在二维码中,检测倾斜角度应用较多的是Hough变换和Radon变换两种方法。\\
Hough变换:利用Hough变换计算倾斜图像角度是广泛使用的一种方法。Hough变换把直线上的点的坐标转换到过点的直线的系数域,对于直线信息明显且干扰小的图像,Hough变换效果比较明显\cite{7}。但在实际使用中,会受到图像周边因素的干扰,或者条码线条比较粗,Hough变换会出现峰值扩散。对高噪声下的二维码,Hough变换效果不佳。并且Hough变换的计算量较大,不能满足实时译码的需要。\\
Radon变换:Radon变换是一种将数字图像变换为在某一指定角度射线方向上投影的方法\cite{7}。在一定范围内对二维码边缘图像求出最大积分的值,然后求出对应于最大积分值的角度,即是条码的旋转角度。该算法能够提高识别的抗噪性,,但是运算速度受到条码旋转角度范围和步进角度影响。\\
不论是Hough变换还是Radon变换,中心思想都是先计算图像倾斜的角度,然后使用点运算将图像旋转纠正。对于仅发生倾斜而没有发生形变的条码纠正效果较好,但对于采集图像过程中易发生的形变问题就束手无策了。

\subsubsection{条码定位}
