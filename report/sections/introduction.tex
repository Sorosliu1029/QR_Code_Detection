\section{引言}
\subsection{问题陈述}
本报告主要依据二维码的编码特点,研究二维码的检测方法。
虽然二维码的编码规范有助于实现快速检测的算法,但是真实场景中,二维码取景时往往伴随着光照不足,摄像设备分辨率低,图像倾斜,变形,扭曲,甚至部分被遮挡的情况。
而我们不单单满足于规范二维码的检测,而是要挑战复杂摄像条件下二维码的检测。\\
具体挑战分为:1. 光照不足引起的图像二值化效果不理想问题,2.摄像设备分辨率低引起的图像边缘检测问题,3.二维码取景时的噪声问题,4.二维码定位模块的位置检测问题,5.多二维码分割问题,6.变形图像的变换问题。

\subsection{应用场合}
由于二维码具有信息编码密度高,信息类型丰富,版本众多,易于定制化等等特点,二维码在生活生产中具有广泛的应用场景。而实现在复杂场景下的二维码检测方法,则能够帮助二维码应用到更多场合中,如光照不足的矿业,使用低端摄像设备的流水线生产行业等等。